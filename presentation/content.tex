%
% This is the content file, see pres.tex or pres-ann.tex
%
%
% Configuration
%
%% Presentation
\usecolortheme{beaver} % Silver + Red
%% Beamer settings
\setbeamersize{
  text margin left=5mm,
  text margin right=5mm
}
\setbeamertemplate{navigation symbols}{} % Remove controls
%% Draft
\usepackage{draftwatermark}
\SetWatermarkText{TODO}
\SetWatermarkScale{2}
\SetWatermarkColor[gray]{0.95}
\setbeamercolor{background canvas}{bg=} % So that beamer does not paint in white on the watermark
%% Encoding
\usepackage[utf8]{inputenc}
%% Symbols
\usepackage{amsmath}
\usepackage{amssymb}
%% Hyperlinks
\usepackage{hyperref} % To use \href and \link
\hypersetup{
  colorlinks=true, % Links will be colored
  urlcolor=darkred % Link color will be a darkish shade of red
}
%% Multiple columns
\usepackage{multicol}
%
% Presentation data
%
%% Beamer
\title{Buildroot}
\subtitle{BASC2020 seminar}
\author{Giacomo Longo}
\institute{University of Genoa}
\date{Q4 2020}
%% Mine
\newcommand{\buildrootLatestVersion}{2020.08.1}
\newcommand{\buildrootLatestVersionAnnouncement}{http://lists.busybox.net/pipermail/buildroot/2020-October/294407.html}
\newcommand{\buildrootLatestVersionDownloadLink}{https://buildroot.org/downloads/buildroot-\buildrootLatestVersion.tar.gz}
\newcommand{\gitRepoUrl}{https://github.com/gabibbo97/basc-buildroot}
%
% Document
%
%%
%% Title para-section
%%
\begin{document}
% Titlepage
\begin{frame}
  \titlepage
\end{frame}
%%
%% Introduction section
%%
\section{BuildRoot}
\subsection{What's BuildRoot}
\begin{frame}
  \frametitle{BuildRoot}
  \begin{figure}
    \begin{center}
      \includegraphics[width=.5\textwidth]{logo-buildroot}
    \end{center}
  \end{figure}
  \begin{center}
    Official website: \url{https://buildroot.org}
  \end{center}
  \begin{itemize}
    \item Born in 2005
    \item Entirely based on \href{https://en.wikipedia.org/wiki/Make_(software)}{makefiles} and \href{https://www.kernel.org/doc/html/latest/kbuild/kconfig-language.html}{kconfig}
    \item Only one goal: \textit{producing root file system images} for \textit{100\% custom Linux systems}
  \end{itemize}
\end{frame}
\begin{frame}
  \frametitle{BuildRoot users}
  The most prominent users of BuildRoot are using it for building:
  \begin{itemize}
    \item IoT devices
    \item Automated factory controllers
    \item Point of sale devices
    \item Car multimedia units
  \end{itemize}
\end{frame}
\subsection{Why BuildRoot}
\begin{frame}
  \frametitle{Why BuildRoot}
  \begin{itemize}
    \item Each buildroot is a 100\% custom Linux "mini-distro"
    \item Buildroot images can be less than 100MB or even 10MB
    \item Complete customization of target architecture and build flags
    \item Multiple compiler / libc / system layout choices
    \item Updated every 3 months {\small current version is \href{\buildrootLatestVersionAnnouncement}{\buildrootLatestVersion}}
    \item Easily extendable
  \end{itemize}
\end{frame}
\begin{frame}
  \frametitle{Why BuildRoot: architecture support}
  \begin{center}
    $ \approx $ 20 architectures supported
  \end{center}
  \begin{multicols}{2}
  \begin{itemize}
    \item ARC {\small LE \& BE}
    \item \textbf{ARM} {\small LE \& BE}
    \item AArch64 {\small LE \& BE}
    \item csky
    \item \textbf{i386}
    \item Microblaze {\small AXI \& Non-AXI}
    \item MIPS {\small LE \& BE}
    \item MIPS64 {\small LE \& BE}
    \item nds32
    \item Nios II
    \item PowerPC
    \item PowerPC64 {\small LE \& BE}
    \item RISCV
    \item SuperH
    \item SPARC
    \item \textbf{x86\_64}
    \item Xtensa
  \end{itemize}
  \end{multicols}
\end{frame}
\subsection{BuildRoot process}
\begin{frame}
  \frametitle{The BuildRoot process}
  \begin{columns}[t] % t == align the first line
    \begin{column}{.5\textwidth}
      \textbf{What the user sees}
      \begin{enumerate}
        \item Create a configuration file
        \item Start the build
        \item Flash the image on the device
      \end{enumerate}
    \end{column}
    \begin{column}{.5\textwidth}
      \textbf{What BuildRoot does}
      \begin{enumerate}
        \item Build a cross compiler on our machine
        \item Resolve the configuration dependencies
        \item Compile from source the requested packages
        \item Assemble an image
      \end{enumerate}
    \end{column}
  \end{columns}
\end{frame}
%%
%% Usage section
%%
\section{Creating some BuildRoots}
\subsection{Prerequisites}
\begin{frame}[fragile]
  \frametitle{Prerequisites}
  \framesubtitle{Packages for an ARM BuildRoot}
  \begin{columns}[t]
    \begin{column}{.5\textwidth}
      \textbf{Ubuntu 20.04}
      \begin{verbatim}
sudo apt-get update
sudo apt-get install -y \
  curl tar \
  make \
  gcc g++ \
  libncurses-dev libssl-dev \
  qemu-user-static \
  qemu-system-arm
      \end{verbatim}
    \end{column}
    \begin{column}{.5\textwidth}
      \textbf{Others} \\
      {\small Binaries needed}
      \begin{description}
        \item[Downloaders] curl \& wget
        \item[Extractor] tar
        \item[Compilers] gcc \& g++
        \item[Libraries] ncurses \& openssl
        \item[Execution] QEMU system for ARM \& QEMU static
      \end{description}
    \end{column}
  \end{columns}
\end{frame}
\begin{frame}
  \frametitle{Preparing our BuildRoot working directory}
  \begin{enumerate}
    \item Clone the repository at \\ \mbox{ \url{\gitRepoUrl} }
    \item Enter the directory
    \item Download BuildRoot from \\ \mbox{ \url{\buildrootLatestVersionDownloadLink} }
    \item Extract the BuildRoot archive
  \end{enumerate}
  \begin{alertblock}{To follow along}
    Ensure you have extracted the BuildRoot archive to \texttt{buildroot-\buildrootLatestVersion}
  \end{alertblock}
\end{frame}
%%%%
%%%% Common phrases library
%%%%
%%%%% Descriptions
\newcommand{\buildOptionsDescription}{
  \textbf{Build Options} define what flags BuildRoot will use to build our executables and libraries. \\
  The compiler cache replaces all compiler invocations with calls to a wrapper compiler called \texttt{ccache} that will avoid compiling the same source twice, don't worry, it's still going to take a lot of time to build everything. \\
  We will be building the entirety of the OS with debugging symbols so gdb will be able to instrument even the system libraries. \\
  The packages are going to be build the maximum debug optimization enabled, so they will be easier to inspect for our debugging tools. \\
  Binary stripping is disabled so all information is preserved.
}
\newcommand{\enteringBuildRootDirectoryDescription}{
  Entering the BuildRoot directory
}
\newcommand{\gefPythonDescription}{
  We will need some Python packages in order to utilize gdb's GEF plugin. \\
  Given that the cross compiler we build is completely indipendent from our system with its own Python installation,
  we need a way to perform this customization at image build time,
  this premade script (\texttt{scripts/gef-python.sh}) will install PIP and the required packages inside the cross compiler environment so we will be able to utilize GEF.
}
\newcommand{\makeCleanDescription}{
  This will cleanup leftover files (if it fails it means there's nothing to remove yet)
}
\newcommand{\makeDefConfigDescription}{
  This will reset the configuration options to BuildRoot defaults
}
\newcommand{\scriptShouldBeExecutableDescription}{
  The script will not be invoked by a shell, but as an executable, so we should ensure correct permissions
}
\newcommand{\sshRootLoginDescription}{
  By default OpenSSH server does not allow access to the \texttt{root} user as a security feature. \\
  In order to be able to simply login to our BuildRoot we need to disable this option.
}
\newcommand{\targetOptionsDescription}{
  \textbf{Target Options} define what target architecture and variant we are targeting. \\
  Our target architecture is ARM little-endian, a commonplace 32bit architecture used in mobile phones. \\
  The variant defines the model of CPU we are targeting, different CPUs might have additional instructions, register layouts and cache hierarchies.
  Including an architecture allows the compiler to optimize register allocation and to use custom instructions. \\
  Here our target variant is the Cortex A7: based on the ARM reference v7 architecture this model is most known for being the heart of the Raspberry Pi 2. \\
  Not all ARM CPUs include hardware support for operation on floating point numbers, here we decide to emulate a processor that supports an optional set of instructions called VFPv4, with support for 16 double precision registers (D16)
}
\newcommand{\toolchainOptionsDescription}{
  \textbf{Toolchain options} define what environment our packages will be linked against. \\
  BuildRoot's choice of default libc is uClibc-ng, a minimal libc built for the most resource-constrained usages. \\
  Given our desktop systems and that the majority of binaries in the wild are linked against the GNU libc, let's use it also inside our BuildRoot. \\
  We need C++ support in order to build some packages without failures. \\
  By requesting a cross gdb we instruct BuildRoot to generate a gdb target at our host architecture that can debug programs written for other architectures. \\
  TUI and Python support are required in order to use GEF.
}
%%%%% Option lists
\newcommand{\targetOptionsList}{
  \item Target options
  \begin{itemize}
    \item Target Architecture = ARM (little endian)
    \item Target Architecture Variant = cortex-A7
    \item Floating point strategy = VFPv4-D16
  \end{itemize}
}
\newcommand{\buildOptionsList}{
  \item Build options
  \begin{itemize}
    \item $\boxtimes$ Enable compiler cache
    \item $\boxtimes$ build packages with debugging symbols
    \item gcc debug level = debug level 3
    \item $\square$ strip target binaries
    \item gcc optimization level = optimize for debugging
  \end{itemize}
}
\newcommand{\toolchainOptionsList}{
  \item Toolchain
  \begin{itemize}
    \item C library = glibc
    \item $\boxtimes$ Enable C++ support
    \item $\boxtimes$ Build cross gdb for the host
    \item $\boxtimes$ TUI support
    \item Python support = Python3
  \end{itemize}
}
%%%%
%%%% ARM cross compiler
%%%%
\subsection{Creating an ARM cross compiler}
\begin{frame}
  \frametitle{Creating an ARM cross compiler}
  \framesubtitle{Initial setup}

  \begin{enumerate}
    \item \texttt{ cd buildroot-\buildrootLatestVersion }
    \item \texttt{ cp ../scripts/gef-python.sh ./gef-python.sh }
    \item \texttt{ chmod +x *.sh }
    \item \texttt{ make distclean }
    \item \texttt{ make defconfig }
  \end{enumerate}

\end{frame}
\only<handout> {
  Steps explaination
  \begin{enumerate}
    \item \enteringBuildRootDirectoryDescription
    \item \gefPythonDescription
    \item \scriptShouldBeExecutableDescription
    \item \makeCleanDescription
    \item \makeDefConfigDescription
  \end{enumerate}
}

\begin{frame}
  \frametitle{Creating an ARM cross compiler}
  \framesubtitle{Configuration options: 1/2}
  \texttt{make menuconfig}
  \begin{itemize}
    \targetOptionsList
    \buildOptionsList
  \end{itemize}
\end{frame}
\only<handout> {
  Options explaination \\
  \targetOptionsDescription \\
  \buildOptionsDescription
}

\begin{frame}
  \frametitle{Creating an ARM cross compiler}
  \framesubtitle{Configuration options: 2/2}
  \begin{itemize}
    \toolchainOptionsList
    \item System configuration
    \begin{itemize}
      \item Custom scripts to run before creating filesystem images = ./gef-python.sh
    \end{itemize}
    \item Filesystem images
    \begin{itemize}
      \item $\square$ tar the root filesystem
    \end{itemize}
    \item Host utilities
    \begin{itemize}
      \item host python3
      \item ssl
    \end{itemize}
  \end{itemize}
\end{frame}
\only<handout> {
  Options explaination \\
  \toolchainOptionsDescription \\
  \gefPythonDescription \\
  We do not need a root filesystem to build a crosscompiler. \\
  We need Python for GEF and OpenSSL in order to install GEF's dependencies using PIP.
}

\begin{frame}
  \frametitle{Creating an ARM cross compiler}
  \framesubtitle{Performing the build}
  \begin{enumerate}
    \item Save the configuration to the default \texttt{.config} path
    \item Download sources with \texttt{make source}
    \item Start the build with \texttt{make sdk}
  \end{enumerate}
\end{frame}
%%%%
%%%% ARM rootfs
%%%%
\subsection{Creating an ARM root filesystem}
\begin{frame}
  \frametitle{Creating an ARM root filesystem}
  \framesubtitle{Initial setup}

  \begin{enumerate}
    \item \texttt{ cd buildroot-\buildrootLatestVersion }
    \item \texttt{ cp ../scripts/gef-python.sh ./gef-python.sh }
    \item \texttt{ chmod +x *.sh }
    \item \texttt{ make distclean }
    \item \texttt{ make defconfig }
  \end{enumerate}

\end{frame}
\only<handout> {
  Steps explaination
  \begin{enumerate}
    \item \enteringBuildRootDirectoryDescription
    \item \gefPythonDescription
    \item \scriptShouldBeExecutableDescription
    \item \makeCleanDescription
    \item \makeDefConfigDescription
  \end{enumerate}
}

\begin{frame}
  \frametitle{Creating an ARM root filesystem}
  \framesubtitle{Configuration options: 1/3}
  \texttt{make menuconfig}
  \begin{itemize}
    \targetOptionsList
    \buildOptionsList
  \end{itemize}
\end{frame}
\only<handout> {
  Options explaination \\
  \targetOptionsDescription \\
  \buildOptionsDescription
}

\begin{frame}
  \frametitle{Creating an ARM root filesystem}
  \framesubtitle{Configuration options: 2/3}
  \begin{itemize}
    \toolchainOptionsList
    \item System configuration
    \begin{itemize}
      \item Custom scripts to run before creating filesystem images = ./gef-python.sh
    \end{itemize}
    \item Target packages
    \begin{itemize}
      \item Debugging, profiling and benchmark
      \begin{itemize}
        \item $\boxtimes$ gdb
        \item $\boxtimes$ full debugger
        \item $\boxtimes$ gdbserver
        \item $\boxtimes$ TUI support
      \end{itemize}
    \end{itemize}
  \end{itemize}
\end{frame}
\only<handout> {
  Options explaination \\
  \toolchainOptionsDescription \\
  \gefPythonDescription \\
  Our root filesystem will also contain gdb and its server.
}

\begin{frame}
  \frametitle{Creating an ARM root filesystem}
  \framesubtitle{Configuration options: 3/3}
  \begin{itemize}
    \item Filesystem images
    \begin{itemize}
      \item $\boxtimes$ tar the root filesystem
    \end{itemize}
    \item Host utilities
    \begin{itemize}
      \item host python3
      \item ssl
    \end{itemize}
  \end{itemize}
\end{frame}
\only<handout> {
  We will package our root filesystem as a tar file. \\
  We need Python for GEF and OpenSSL in order to install GEF's dependencies using PIP.
}

\begin{frame}
  \frametitle{Creating an ARM root filesystem}
  \framesubtitle{Performing the build}
  \begin{enumerate}
    \item Save the configuration to the default \texttt{.config} path
    \item Download sources with \texttt{make source}
    \item Start the build with \texttt{make}
  \end{enumerate}
\end{frame}
%%%%
%%%% ARM bootable rootfs
%%%%
\subsection{Creating a bootable ARM root filesystem}
\begin{frame}
  \framesubtitle{Initial setup}

  \begin{enumerate}
    \item \texttt{ cd buildroot-\buildrootLatestVersion }
    \item \texttt{ cp ../kconfigs/virtio.kconfig ./virtio.kconfig }
    \item \texttt{ cp ../scripts/gef-python.sh ./gef-python.sh }
    \item \texttt{ cp ../scripts/enable-ssh-root-login.sh ./enable-ssh-root-login.sh }
    \item \texttt{ chmod +x *.sh }
    \item \texttt{ make distclean }
    \item \texttt{ make defconfig }
  \end{enumerate}

\end{frame}
\only<handout> {
  Steps explaination
  \begin{enumerate}
    \item \enteringBuildRootDirectoryDescription
    \item In order to use our built rootfs inside QEMU we need to add some Kernel modules,
      this file contains the kernel configuration options that should be set to support smooth virtualization under QEMU
    \item \gefPythonDescription
    \item \sshRootLoginDescription
    \item \scriptShouldBeExecutableDescription
    \item \makeCleanDescription
    \item \makeDefConfigDescription
  \end{enumerate}
}

\begin{frame}
  \frametitle{Creating a bootable ARM root filesystem}
  \framesubtitle{Configuration options: 1/3}
  \texttt{make menuconfig}
  \begin{itemize}
    \targetOptionsList
    \buildOptionsList
  \end{itemize}
\end{frame}
\only<handout> {
  Options explaination \\
  \targetOptionsDescription \\
  \buildOptionsDescription
}

\begin{frame}
  \frametitle{Creating a bootable ARM root filesystem}
  \framesubtitle{Configuration options: 2/3}
  \begin{itemize}
    \toolchainOptionsList
    \item System configuration
    \begin{itemize}
      \item System hostname = BASC2020
      \item System banner = Welcome to BASC2020 Buildroot
      \item Root password = BASC2020
      \item Network interface to configure through DHCP = eth0
      \item Custom scripts to run before creating filesystem images = ./enable-ssh-root-login.sh ./gef-python.sh
    \end{itemize}
  \end{itemize}
\end{frame}
\only<handout> {
  Options explaination \\
  \toolchainOptionsDescription \\
  The system customization of hostname / banner is simply aestetic. \\
  The root password has to be set in order to be able to login. \\
  We choose to automatically configure the first ethernet network interface in order to avoid bothering with networking settings after the rootfs will be booted in QEMU. \\
  \sshRootLoginDescription \\
  \gefPythonDescription
}

\begin{frame}
  \frametitle{Creating a bootable ARM root filesystem}
  \framesubtitle{Configuration options: 3/3}
  \begin{itemize}
    \item Target packages
    \begin{itemize}
      \item Debugging, profiling and benchmark
      \begin{itemize}
        \item $\boxtimes$ gdb
        \item $\boxtimes$ full debugger
        \item $\boxtimes$ gdbserver
        \item $\boxtimes$ TUI support
        \item $\boxtimes$ ltrace
        \item $\boxtimes$ strace
        \item $\boxtimes$ valgrind
      \end{itemize}
      \item Networking applications
      \begin{itemize}
        \item $\boxtimes$ openssh
        \item $\square$ client
        \item $\boxtimes$ key utilities
      \end{itemize}
    \end{itemize}
    \item Filesystem images
    \begin{itemize}
      \item $\boxtimes$ ext2/3/4 root filesystem
      \begin{itemize}
        \item exact size = 512M
      \end{itemize}
      \item $\square$ tar the root filesystem
    \end{itemize}
    \item Host utilities
    \begin{itemize}
      \item host python3
      \item ssl
    \end{itemize}
  \end{itemize}
\end{frame}
\only<handout> {
  Our root filesystem will also contain gdb and its server. \\
  Given that we will be running an actual OS, ltrace, strace and valgrind will be supported and can be used. \\
  An OpenSSH server will allow us to connect and sync files from/to the rootfs. \\
  We will package our root filesystem as an ext image. \\
  We need Python for GEF and OpenSSL in order to install GEF's dependencies using PIP.
}

\begin{frame}
  \frametitle{Creating an ARM root filesystem}
  \framesubtitle{Performing the build}
  \begin{enumerate}
    \item Save the configuration to the default \texttt{.config} path
    \item Download sources with \texttt{make source}
    \item Start the build with \texttt{make}
  \end{enumerate}
\end{frame}
%%%%
%%%% Customization
%%%%
\subsection{Customizing our images}
\begin{frame}[label={Customizing our images}]
  \frametitle{Customizing our images}
  \framesubtitle{Build time overlay}
  \begin{itemize}
    \item Create a directory
    \item Add \texttt{BR2\_ROOTFS\_OVERLAY=my-overlay} to \texttt{.config}
    \item Rebuild using \texttt{make}
    \item The structure of \texttt{my-overlay} will be copied to the rootfs
  \end{itemize}
  \begin{block}{How to specify multiple overlays}
    Multiple overlays can be specified by separating them with spaces in the \texttt{BR2\_ROOTFS\_OVERLAY} directive
  \end{block}
\end{frame}
\begin{frame}
  \frametitle{Customizing our images}
  \framesubtitle{Build time script}
  Add \texttt{BR2\_ROOTFS\_POST\_BUILD\_SCRIPT=my-script.sh} to \texttt{.config} \\
  Available environment variables inside:
  \begin{table}
    \begin{center}
      \begin{tabular}{ll}
        \texttt{BR2\_CONFIG} & path of \texttt{.config} \\
        \texttt{HOST\_DIR} & path of \texttt{output/host} \\
        \texttt{STAGING\_DIR} & path of \texttt{output/staging} \\
        \texttt{TARGET\_DIR} & path of \texttt{output/target} \\
        \texttt{BUILD\_DIR} & path of \texttt{output/build} \\
        \texttt{BINARIES\_DIR} & path of \texttt{output/images} \\
        \texttt{BASE\_DIR} & path of \texttt{output} \\
      \end{tabular}
    \end{center}
  \end{table}
  \begin{block}{How to specify multiple scripts}
    Multiple scripts can be specified by separating them with spaces in the \texttt{BR2\_ROOTFS\_POST\_BUILD\_SCRIPT} directive
  \end{block}
\end{frame}
\begin{frame}
  \frametitle{Customizing our images}
  \framesubtitle{Editing the target directory}
  \begin{enumerate}
    \item Add your files to the \texttt{output/target} directory
    \item Rebuild using \texttt{make}
  \end{enumerate}
  \begin{alertblock}{Warning}
    Your files might be rewritten / deleted by buildroot
  \end{alertblock}
\end{frame}
\begin{frame}
  \frametitle{Customizing our images}
  \framesubtitle{D.I.Y. approach}
  \begin{enumerate}
    \item Unpack your rootfs (with \texttt{tar -xzf} for instance)
    \item Perform your modifications
    \item Repack your rootfs (with \texttt{tar -cf} for instance)
  \end{enumerate}
\end{frame}
%%
%% Using BuildRoot section
%%
\section{Using our BuildRoot}
\subsection{Producing binaries for the target}
\begin{frame}
  \frametitle{Using the cross-compiler}
  \begin{enumerate}
    \item Extract the cross-compiler
    \item Run \texttt{relocate-sdk.sh}
    \item Edit your \texttt{ \$PATH } variable: \texttt{export PATH="\${PATH}:\${PWD}/bin"}
    \item You can invoke your cross compiler with commands like \texttt{arm-buildroot-linux-gnueabihf-\textit{<COMMAND NAME>}} \\
    \textbf{Notable entries}
    \begin{itemize}
      \item \texttt{arm-buildroot-linux-gnueabihf-gcc}
      \item \texttt{arm-buildroot-linux-gnueabihf-gdb}
      \item \texttt{arm-buildroot-linux-gnueabihf-nm}
    \end{itemize}
  \end{enumerate}
  \begin{alertblock}{Improving gdb with library simbols}
    See the section \hyperlink{important}{\beamergotobutton{Using gdb}}
  \end{alertblock}
\end{frame}

\subsection{Running dynamic executables}
\begin{frame}[fragile]
  \frametitle{Running dynamic executables in Docker}
  \begin{verbatim}
sudo docker import rootfs.tar basc-buildroot
sudo docker run --rm -it \
  --volume "$(which qemu-arm-static):/bin/qemu-arm-static" \
  --volume "${PWD}/:/host" \
  --entrypoint /bin/qemu-arm-static \
  --workdir "/host" \
  basc-buildroot \
  /bin/sh
  \end{verbatim}
\end{frame}
\only<handout> {
  Docker supports importing .tar files as images for containers. \\
  That docker import command is equivalent to pulling an image consisting of a single layer comprised of our rootfs. \\
  We will run the executables in QEMU by leveraging the static binary provided by the host, in this configuration
  most dynamic executables will work but some calls to fork/exec syscalls and any tracing request will fail.
}

\begin{frame}[fragile]
  \frametitle{Running dynamic executables with systemd-nspawn}
  \begin{verbatim}
mkdir -p basc-rootfs
tar -xf rootfs.tar -C basc-rootfs
cp -f "$(which qemu-arm-static)" \
  basc-rootfs/bin/qemu-arm-static
sudo systemd-nspawn \
  --register=no \
  -D basc-rootfs \
  /bin/qemu-arm-static /bin/sh
  \end{verbatim}
  \begin{alertblock}{Package needed}
  You might need to install the package \texttt{systemd-container}
  \end{alertblock}
\end{frame}
\only<handout> {
  Systemd-nspawn is like chroot, with some nifty additions (for istance it automatically mounts the pseudo filesystems for us),
   the --register=no is due to the fact that we want an ephemeral machine and not a system service. \\
  We will run the executables in QEMU by leveraging the static binary provided by the host, in this configuration
  most dynamic executables will work but some calls to fork/exec syscalls and any tracing request will fail.
}

\subsection{Performing dynamic analysis}
\begin{frame}[fragile]
  \frametitle{Booting the rootfs}
  \footnotesize
  \begin{verbatim}
qemu-system-arm \
  -machine virt \
  -cpu cortex-a7 \
  -smp 4 -m 4096 \
  -kernel zImage \
  -device virtio-blk-device,drive=rootfs \
  -drive file=rootfs.ext2,if=none,format=raw,id=rootfs \
  -append "console=ttyAMA0,115200 rootwait root=/dev/vda" \
  -netdev user,id=user0,hostfwd=tcp::2222-:22,hostfwd=tcp::1234-:1234 \
  -device virtio-net-device,netdev=user0 \
  -serial stdio \
  -display none
  \end{verbatim}
\end{frame}
\only<handout> {
  QEMU system emulation requires a lot of arguments:
  \begin{description}
    \item[machine] QEMU can emulate a specific machine or a more "generic" one,
      virt is the recommended hardware configuration for general purpose ARM emulation. \\
      The usage of a preexisting machine can also limit the availability of hardware, lock the number of cpus or cap the maximum memory available. \\
      ARM machines do not have the equivalent of BIOS / EFI standards and so there is no hardware automatic discovery, hardware support has to be precompiled into the kernel with device tree files. \\
      In order to avoid limitations we have explicitly compiled in VIRTIO paravirtualization drivers to make the system aware of it being emulated.
    \item[cpu] This sets which instructions are available to the guest,
      unincluded instructions will trap the hypervisor giving an effect similar to issuing an illegal instruction on a real cpu
    \item[smp] Sets the number of processors
    \item[m] Sets the amount of RAM in megabytes
    \item[kernel] The image has no bootloader in it so we need QEMU to load the kernel for us
    \item[device virtio-blk-device] This creates a virtual hard disk, supported by the virtio drivers we included in the kernel
    \item[drive] This binds the disk drive to the rootfs we specified
    \item[append] This will be the kernel command line, specifying a serial console and the root device location
    \item[netdev] This sets up the virtual network card and forward host port 2222 to port 22 of guest and port 1234 of host to port 1234 of guest
    \item[device virtio-net-device] This will create a virtual network card, supported by the virtio drivers we included in the kernel
    \item[serial stdio] This will allow us to interact with the console on the calling terminal
    \item[display none] We do not have a display nor the rootfs has support for one
  \end{description}
}

% \begin{frame}[fragile]
%   \frametitle{Tips and tricks}
%   \framesubtitle{Enable SSH root login}

%   \textit{In your BuildRoot shell}
%   \begin{enumerate}
%     \item Edit \texttt{/etc/ssh/sshd\_config} to include \texttt{PermitRootLogin yes}
%     \item Find the \texttt{openssh} server process id number
%     \begin{verbatim}
% # ps | grep 'sshd'
%   138 root     sshd: /usr/sbin/sshd [listener] 0 of 10-100 startups
%   158 root     grep sshd
%     \end{verbatim}
%     \item Reload the \text{openssh} process
%     \begin{verbatim}
% # kill -s HUP 138
%     \end{verbatim}
%   \end{enumerate}
%   \begin{block}{Automating the process}
%     See the section on \hyperlink{important}{\beamergotobutton{Customizing our images}}
%   \end{block}
% \end{frame}

\begin{frame}[fragile]
  \frametitle{Tips and tricks}
  \framesubtitle{Using SSH}
  \begin{center}
    \textbf{Opening an SSH session}
  \end{center}
  \begin{verbatim}
ssh \
  -o UserKnownHostsFile=/dev/null \
  -o StrictHostKeyChecking=no \
  -p 2222 root@localhost
  \end{verbatim}

  \begin{center}
    \textbf{Sharing a folder}
  \end{center}
  \begin{verbatim}
mkdir -p guest-os-ssh
sshfs root@localhost:/ ./guest-os-ssh \
  -f \
  -o port=2222 \
  -o reconnect \
  -o UserKnownHostsFile=/dev/null \
  -o StrictHostKeyChecking=no
  \end{verbatim}
\end{frame}

\begin{frame}
  \frametitle{Using ltrace and strace}
  \framesubtitle{What did you expect?}
  \begin{itemize}
    \item \texttt{ltrace} and \texttt{strace} do work as expected
    \item Can only be performed on QEMU \textbf{system} emulation
  \end{itemize}
\end{frame}
\begin{frame}
  \frametitle{Using gdb}
  \textbf{On the guest} \\
  \texttt{gdbserver :1234 \textit{command to debug}} \\
  \textbf{On the host} (\textit{From the cross-compiler extracted folder}) \\
  \texttt{
    \small
    bin/{\tiny arm-buildroot-linux-gnueabihf-}gdb $\backslash$ \\
      \quad -x {\tiny arm-buildroot-linux-gnueabihf}/sysroot/usr/share/buildroot/gdbinit $\backslash$ \\
      \quad \textit{executable name}
  }\\
  \textbf{On the host gdb shell} attach with \texttt{target remote localhost:1234}
\end{frame}
\only<handout> {
  The \texttt{-x} argument points gdb to an additional \texttt{gdbinit}. \\
  The only command inside is \texttt{set sysroot} that will allow gdb to access libraries pertaining to the guest
}

\end{document}

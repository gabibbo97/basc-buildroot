%
% Configuration
%
%% Presentation
\documentclass{beamer}
\usecolortheme{beaver} % Silver + Red
%% Encoding
\usepackage[utf8]{inputenc}
%% Symbols
\usepackage{amsmath}
%% Hyperlinks
\usepackage{hyperref} % To use \href and \link
\hypersetup{
  colorlinks=true, % Links will be colored
  urlcolor=darkred % Link color will be a darkish shade of red
}
%% Multiple columns
\usepackage{multicol}
%
% Presentation data
%
%% Beamer
\title{Buildroot}
\subtitle{BASC2020 seminar}
\author{Giacomo Longo}
\institute{University of Genoa}
\date{TODO}
%% Mine
\newcommand{\buildrootLatestVersion}{2020.08.1}
\newcommand{\buildrootLatestVersionAnnouncement}{http://lists.busybox.net/pipermail/buildroot/2020-October/294407.html}
\newcommand{\buildrootLatestVersionDownloadLink}{https://buildroot.org/downloads/buildroot-\buildrootLatestVersion.tar.gz}
%
% Document
%
%%
%% Title para-section
%%
\begin{document}
% Titlepage
\begin{frame}
  \titlepage
\end{frame}
% Table of contents
\begin{frame}
  \frametitle{Table of contents}
  \tableofcontents
\end{frame}
%%
%% Introduction section
%%
\section{BuildRoot}
\subsection{What's BuildRoot}
\begin{frame}
  \frametitle{BuildRoot}
  \begin{figure}
    \begin{center}
      \includegraphics[width=.5\textwidth]{logo-buildroot}
    \end{center}
  \end{figure}
  \begin{center}
    Official website: \url{https://buildroot.org}
  \end{center}
  \begin{itemize}
    \item Born in 2005
    \item Entirely based on \href{https://en.wikipedia.org/wiki/Make_(software)}{makefiles} and \href{https://www.kernel.org/doc/html/latest/kbuild/kconfig-language.html}{kconfig}
    \item Only one goal: \textit{producing root file system images} for \textit{100\% custom Linux systems}
  \end{itemize}
\end{frame}
\begin{frame}
  \frametitle{BuildRoot users}
  The most prominent users of BuildRoot are using it for building:
  \begin{itemize}
    \item IoT devices
    \item Automated factory controllers
    \item Point of sale devices
    \item Car multimedia units
  \end{itemize}
\end{frame}
\subsection{Why BuildRoot}
\begin{frame}
  \frametitle{Why BuildRoot}
  \begin{itemize}
    \item Each buildroot is a 100\% custom Linux "mini-distro"
    \item Buildroot images can be less than 100MB or even 10MB
    \item Complete customization of target architecture and build flags
    \item Multiple compiler / libc / system layout choices
    \item Updated every 3 months {\small current version is \href{\buildrootLatestVersionAnnouncement}{\buildrootLatestVersion}}
    \item Easily extendable
  \end{itemize}
\end{frame}
\begin{frame}
  \frametitle{Why BuildRoot: architecture support}
  \begin{center}
    $ \approx $ 20 architectures supported
  \end{center}
  \begin{multicols}{2}
  \begin{itemize}
    \item ARC {\small LE \& BE}
    \item \textbf{ARM} {\small LE \& BE}
    \item AArch64 {\small LE \& BE}
    \item csky
    \item \textbf{i386}
    \item Microblaze {\small AXI \& Non-AXI}
    \item MIPS {\small LE \& BE}
    \item MIPS64 {\small LE \& BE}
    \item nds32
    \item Nios II
    \item PowerPC
    \item PowerPC64 {\small LE \& BE}
    \item RISCV
    \item SuperH
    \item SPARC
    \item \textbf{x86\_64}
    \item Xtensa
  \end{itemize}
  \end{multicols}
\end{frame}
\subsection{BuildRoot process}
\begin{frame}
  \frametitle{The BuildRoot process}
  \begin{columns}[t] % t == align the first line
    \begin{column}{.5\textwidth}
      \textbf{What the user sees}
      \begin{enumerate}
        \item Create a configuration file
        \item Start the build
        \item Flash the image on the device
      \end{enumerate}
    \end{column}
    \begin{column}{.5\textwidth}
      \textbf{What BuildRoot does}
      \begin{enumerate}
        \item Build a cross compiler on our machine
        \item Resolve the configuration dependencies
        \item Compile from source the requested packages
        \item Assemble an image
      \end{enumerate}
    \end{column}
  \end{columns}
\end{frame}
%%
%% Usage section
%%
\section{Creating some BuildRoots}
\subsection{Prerequisites}
\begin{frame}[fragile]
  \frametitle{Prerequisites}
  \framesubtitle{Packages for an ARM BuildRoot}
  \begin{columns}[t]
    \begin{column}{.5\textwidth}
      \textbf{Ubuntu 20.04}
      \begin{verbatim}
sudo apt-get update
sudo apt-get install -y \
  curl tar \
  make \
  gcc g++ \
  libncurses-dev libssl-dev \
  qemu-user-static \
  qemu-system-arm
      \end{verbatim}
    \end{column}
    \begin{column}{.5\textwidth}
      \textbf{Others} \\
      {\small Binaries needed}
      \begin{description}
        \item[Downloaders] curl \& wget
        \item[Extractor] tar
        \item[Compilers] gcc \& g++
        \item[Libraries] ncurses \& openssl
        \item[Execution] QEMU system for ARM \& QEMU static 
      \end{description}
    \end{column}
  \end{columns}
\end{frame}
\begin{frame}[fragile]
  \frametitle{Obtaining BuildRoot}
  \small
  Download from: \\
  \url{\buildrootLatestVersionDownloadLink} \\
  Extract with \texttt{tar -xzf} \\
  Your BuildRoot files will be in \texttt{buildroot-\buildrootLatestVersion}
\end{frame}
%%%%
%%%% ARM cross compiler
%%%%
\subsection{Creating an ARM cross compiler}
\begin{frame}
  \frametitle{Creating an ARM cross compiler}
  \begin{center}
    TODO
  \end{center}
\end{frame}
%%%%
%%%% ARM rootfs
%%%%
\subsection{Creating an ARM root filesystem}
\begin{frame}
  \frametitle{Creating an ARM root filesystem}
  \begin{center}
    TODO
  \end{center}
\end{frame}
%%%%
%%%% ARM bootable rootfs
%%%%
\subsection{Creating a bootable ARM root filesystem}
\begin{frame}
  \frametitle{Creating a bootable ARM root filesystem}
  \begin{center}
    TODO
  \end{center}
\end{frame}
\subsection{Customizing our images}
\begin{frame}
  \frametitle{Customizing our images}
  \framesubtitle{Build time overlay}
  \begin{itemize}
    \item Create a directory
    \item Add \texttt{BR2\_ROOTFS\_OVERLAY=my-overlay} to \texttt{.config}
    \item Rebuild using \texttt{make}
    \item The structure of \texttt{my-overlay} will be copied to the rootfs
  \end{itemize}
  \begin{block}{How to specify multiple overlays}
    Multiple overlays can be specified by separating them with spaces in the \texttt{BR2\_ROOTFS\_OVERLAY} directive
  \end{block}
\end{frame}
%%%%
%%%% Customization
%%%%
\begin{frame}
  \frametitle{Customizing our images}
  \framesubtitle{Build time script}
  Add \texttt{BR2\_ROOTFS\_POST\_BUILD\_SCRIPT=my-script.sh} to \texttt{.config} \\
  Available environment variables inside:
  \begin{table}
    \begin{center}
      \begin{tabular}{ll}
        \texttt{BR2\_CONFIG} & path of \texttt{.config} \\
        \texttt{HOST\_DIR} & path of \texttt{output/host} \\
        \texttt{STAGING\_DIR} & path of \texttt{output/staging} \\
        \texttt{TARGET\_DIR} & path of \texttt{output/target} \\
        \texttt{BUILD\_DIR} & path of \texttt{output/build} \\
        \texttt{BINARIES\_DIR} & path of \texttt{output/images} \\
        \texttt{BASE\_DIR} & path of \texttt{output} \\
      \end{tabular}
    \end{center}
  \end{table}
  \begin{block}{How to specify multiple scripts}
    Multiple scripts can be specified by separating them with spaces in the \texttt{BR2\_ROOTFS\_POST\_BUILD\_SCRIPT} directive
  \end{block}
\end{frame}
\begin{frame}
  \frametitle{Customizing our images}
  \framesubtitle{Editing the target directory}
  \begin{enumerate}
    \item Add your files to the \texttt{output/target} directory
    \item Rebuild using \texttt{make}
  \end{enumerate}
  \begin{alertblock}{Warning}
    Your files might be rewritten / deleted by buildroot
  \end{alertblock}
\end{frame}
\begin{frame}
  \frametitle{Customizing our images}
  \framesubtitle{D.I.Y. approach}
  \begin{enumerate}
    \item Unpack your rootfs (with \texttt{tar -xzf} for instance)
    \item Perform your modifications
    \item Repack your rootfs (with \texttt{tar -cf} for instance)
  \end{enumerate}
\end{frame}
%%
%% Using BuildRoot section
%%
\section{Using our BuildRoot}
\subsection{Running dynamic executables}
\begin{frame}
  \frametitle{Running dynamic executables in Docker}
  \begin{center}
    TODO
  \end{center}
\end{frame}
\begin{frame}
  \frametitle{Running dynamic executables with systemd-nspawn}
  \begin{center}
    TODO
  \end{center}
\end{frame}
\subsection{Performing dynamic analysis}
\begin{frame}
  \frametitle{Tips and tricks}
  \begin{center}
    TODO
  \end{center}
\end{frame}
\begin{frame}
  \frametitle{Using ltrace}
  \begin{center}
    TODO
  \end{center}
\end{frame}
\begin{frame}
  \frametitle{Using strace}
  \begin{center}
    TODO
  \end{center}
\end{frame}
\begin{frame}
  \frametitle{Using gdb}
  \begin{center}
    TODO
  \end{center}
\end{frame}

\end{document}
